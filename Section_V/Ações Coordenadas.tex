\clearpage
\sffamily
{\bfseries\color[rgb]{0.4,0.4,0.4}Desafios de Comunicação - Ações Coordenadas}
\phantomsection
\addcontentsline{toc}{subsection}{Ações Coordenadas}

\bigskip

{\bfseries Objetivo}

\headlinebox

O objetivo deste desafio é testar a capacidade dos robôs de se comunicarem e realizarem ações coordenadas em resposta a comandos específicos.

\bigskip

{\bfseries Regras do Desafio}

\headlinebox

\begin{itemize}
	\item Haverá dois elementos em campo: um robô com a bola e um dispositivo de comunicação, que pode ser outro robô ou um dispositivo dedicado.
	\item A comunicação entre os dispositivos deve ser realizada exclusivamente via Wi-Fi da categoria, que será disponibilizado pela organização. O uso de cabos ou outros meios de comunicação não será permitido.
	\item A equipe poderá escolher até cinco ações para o robô em campo realizar, como chutar a bola, andar, ou acenar.
	\item Durante o desafio, o avaliador mostrará ao dispositivo de comunicação objetos determinados pela equipe. O robô em campo deve, então, realizar a ação correspondente ao comando recebido.
\end{itemize}

\bigskip

{\bfseries Critérios de Avaliação}

\headlinebox

\begin{itemize}
	\item Cada ação realizada corretamente dará à equipe 10 pontos.
	\item A ordem de execução das ações será determinada pelo avaliador no momento do desafio.
	\item Ambos os robôs devem estar conectados ao Game Controller e devem aparecer como robôs no sistema. Caso o Game Controller não seja utilizado, haverá uma penalidade de 15\% na pontuação.
\end{itemize}