\clearpage
\sffamily
{\bfseries\color[rgb]{0.4,0.4,0.4}Desafios de Comunicação - Navegação com Visão}
\phantomsection
\addcontentsline{toc}{subsection}{Navegação com Visão}

\bigskip

{\bfseries Objetivo}

\headlinebox

Este desafio avalia a capacidade dos robôs de usar comunicação e visão para guiar o Robô A até a bola no menor tempo possível.

\bigskip

{\bfseries Descrição do Desafio}

\headlinebox

\begin{itemize}
	\item O Robô A deve ser guiado para tocar a bola. A bola será posicionada pelo avaliador em qualquer lugar dentro de um raio de 3 metros do Robô A.
	\item Após a bola ser posicionada, a equipe pode posicionar o Robô B (observador) em qualquer lugar do campo.
	\item O Robô B pode ser um robô, ou um dispositivo usado no desafio de visão, que não precisa ser um robô físico completo.
	\item O Robô B deve fornecer informações ao Robô A sobre a posição da bola e o caminho a seguir.
	\item A comunicação entre os robôs deve ser realizada exclusivamente via Wi-Fi da categoria; não é permitido o uso de cabos ou outros meios de comunicação.
\end{itemize}

\bigskip

{\bfseries Regras do Desafio}

\headlinebox
\begin{itemize}
	\item O Robô A deve apenas tocar a bola para completar o desafio.
	\item O tempo começa a ser contado quando o Robô A se move e é interrompido quando o Robô A toca a bola.
	\item Cada equipe terá 3 (três) minutos para realizar o desafio, com até 3 (três) tentativas permitidas.
\end{itemize}

\bigskip

{\bfseries Critérios de Avaliação}

\headlinebox

\begin{itemize}
	\item A pontuação será baseada no tempo total que o Robô A leva para tocar a bola, desde o início do movimento até o contato com a bola.
	\item A equipe que conseguir tocar a bola no menor tempo receberá a maior pontuação.
	\item Se o Robô A não tocar a bola, o tempo gasto e as ações realizadas serão considerados para a pontuação final.
\end{itemize}