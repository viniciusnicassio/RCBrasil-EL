\clearpage
\sffamily
{\bfseries\color[rgb]{0.4,0.4,0.4}
	PROCEDIMENTOS PARA DETERMINAR O VENCEDOR DA COMPETIÇÃO}
\phantomsection
\addcontentsline{toc}{subsection}{Procedimentos para determinar o Vencedor da
	Competição}

\bigskip

Como a competição Entry Level tem como objetivo principal a promoção da
pesquisa e do desenvolvimento de robôs humanoides, além de incentivar novas
equipes na categoria Humanoid Soccer League, os candidatos a vencedores são as
equipes que participam exclusivamente da competição Entry Level. As equipes que
participam tanto da competição Entry Level quanto da Humanoid Soccer League não
são elegíveis para o título de campeão da competição Entry Level; essas equipes
receberão apenas um certificado de participação e terão seus resultados
listados na tabela de classificação.

Equipes ganhadoras de versões anteriores da competição Entry Level e avaliadas
como aptas a participar da Humanoid Soccer League pelo Comitê Técnico da Entry
Level não são elegíveis para o título de campeão da competição Entry Level.

Para as equipes elegíveis da competição Entry Level, o vencedor será
determinado da seguinte forma:

\begin{itemize}
	\item Uma média dos resultados obtidos em cada desafio será calculada
	      de
	      acordo com a área do desafio (Visão, Controle, e Comunicação)
	\item Após o cálculo da média, será feita uma somatória dos resultados
	      obtidos em cada categoria
	\item A equipe com a maior somatória será declarada vencedora da
	      competição
	      Entry Level
	\item Em caso de empate, a equipe com a maior quantidade de desafios
	      completados será declarada vencedora.
	\item Se o empate persistir, a equipe com o menor tempo total de
	      conclusão
	      dos desafios será declarada vencedora.
	\item Se o empate ainda persistir, uma moeda será lançada pela Comissão
	      Técnica.
\end{itemize}
