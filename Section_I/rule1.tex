\clearpage
\sffamily
{\bfseries\color[rgb]{0.4,0.4,0.4}
Regra 1 – O Campo dos Desafios}
\phantomsection
\addcontentsline{toc}{subsection}{Regra 1 – O Campo dos Desafios}

\bigskip
{\bfseries Superfície do campo}

\headlinebox
Os desafios serão executados no mesmo campo da competição RoboCup Brasil Humanoid Soccer League, cuja superfície é de grama artificial com altura aproximada de 30 mm.

\bigskip

(Modificações: Caso os organizadores locais não consigam encontrar uma grama com altura aproximada de 30 mm, uma grama com altura menor pode ser utilizada sem aviso prévio às equipes.)

\bigskip
{\sffamily
A cor das superfícies artificiais deve ser verde.}

\bigskip
{\bfseries
Marcações de campo}

\headlinebox

O campo de jogo deve ser retangular e marcado com linhas. Essas linhas pertencem às áreas das quais são limites.

\bigskip

As duas linhas de limite mais longas são chamadas de linhas de toque. As duas linhas mais curtas são chamadas de linhas de gol.

\bigskip

O campo de jogo é dividido em duas metades por uma linha intermediária, que une os pontos médios das duas linhas laterais.

\bigskip

A marca central é indicada no ponto médio da linha intermediária.
Um círculo está marcado ao seu redor.

\bigskip

{\textbf{Dimensões}}

\headlinebox

As dimenções do campo estão descritos na \href{https://cbr.robocup.org.br/wp-content/uploads/2024/04/LARC2024.pdf}{\textcolor[rgb]{0,0,0.5 }{Regra da Competição da RoboCup Brasil Humanoid Soccer League}}.