\clearpage
\sffamily
{\bfseries\color[rgb]{0.4,0.4,0.4}
Regra 5 – O Avaliador}
\phantomsection
\addcontentsline{toc}{subsection}{Regra 5 – O Avaliador}

\bigskip

{\bfseries À autoridade do avaliador}

\headlinebox

Cada desafio é controlado por um avaliador que tem autoridade total para fazer cumprir as regras do desafio para o qual foi nomeado. As decisões serão tomadas da melhor maneira possível, de acordo com as regras do desafio e o espírito fraternal, baseando-se na opinião ou programação dos avaliadores, que têm o poder de tomar as medidas apropriadas dentro da estrutura das regras do desafio.

Os jogos são supervisionados pela Comissão Técnica da liga, que garante que os jogadores e o ambiente estão de acordo com as regras do desafio, podendo sancionar comportamentos antidesportivos das equipes.

\bigskip

{\bfseries Poderes e deveres}

\headlinebox

O avaliador:

\begin{itemize}
    \item aplica as regras do desafio
    \item garante que qualquer bola usada atenda aos requisitos da Lei 2
    \item garante que o equipamento dos jogadores atenda aos requisitos da regra 4
    \item atua como cronometrista e mantém um registro do desafio e da pontuação
    \item interrompe, suspende ou abandona o desafio, a seu critério, por qualquer infração às regras
    \item interrompe, suspende ou abandona o desafio devido a interferência externa de qualquer tipo
    \item permite que um desafio continue até que ele seja concluido mesmo que um jogador estiver, na sua opinião, apenas ligeiramente lesionado
    \item toma medidas contra os oficiais da equipe que não se comportam de maneira responsável e pode, a seu critério, expulsá-los do campo do desafio e de seu entorno imediato
    \item garante que nenhuma pessoa não autorizada entre no campo do desafio
\end{itemize}

\bigskip

{\bfseries Decisões do Avaliador}

\headlinebox

As decisões do avaliador sobre os fatos relacionados ao desafio, incluindo o tempo cronometrado e a pontuação obtida, são finais. 

\bigskip

Na competição física, o avaliador só pode alterar uma decisão ao perceber que está incorreta ou, a seu critério, a conselho de um avaliador assistente, desde que não tenha reiniciado ou encerrado o desafio.