\clearpage
\sffamily
{\bfseries\color[rgb]{0.4,0.4,0.4}{Regra 8 – O Método de Pontuação} }
\phantomsection
\addcontentsline{toc}{subsection}{Regra 8 – O Método de Pontuação}

\bigskip

{\bfseries Tempo de conclusão do desafio}

\headlinebox

Uma das maneiras de pontuar é concluir o desafio no menor tempo possível. O tempo de conclusão do desafio é o tempo decorrido entre o início do desafio e o momento em que o robô atinge a posição final. O tempo de conclusão do desafio é medido em minutos e segundos.

A pontuação é calculada com base no tempo de conclusão do desafio. A pontuação é calculada de acordo com a seguinte fórmula:
$$
\text{Pontuação} = \frac{10 \text{ minutos} - \text{Tempo de conclusão do desafio}}{10 \text{ minutos}} \times P_{max}
$$
Sendo $P_{max}$ a pontuação máxima do desafio.

\bigskip

{\bfseries Pontuação}

\headlinebox

Outra maneira de pontuar é atingir um ou mais objetivo específico durante um desafio. Nessa caso a pontuação do desafio será a soma das pontuações de cada objetivo atingido.