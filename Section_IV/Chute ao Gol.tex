\clearpage
\sffamily
{\bfseries\color[rgb]{0.4,0.4,0.4}Desafio de Controle - Chute ao Gol}
\phantomsection
\addcontentsline{toc}{subsection}{Chute ao Gol}

\bigskip

{\bfseries Objetivo}

\headlinebox

O objetivo deste desafio é avaliar a capacidade do robô de chutar uma bola em direção ao gol. A bola será posicionada na marca do pênalti, e o robô terá 3 minutos para realizar o chute e tentar fazer o gol. Não haverá goleiro durante este desafio.

\bigskip

{\bfseries Regras do Desafio}

\headlinebox

\begin{itemize}
	\item O robô será posicionado a uma distância de 50 centímetros da bola.
	\item O robô poderá escolher entre chutar diretamente a bola em direção ao gol ou conduzir a bola até o gol antes de finalizar o chute.
	\item Cada robô dispõe de 3 (três) tentativas para tentar fazer o gol.
\end{itemize}

\bigskip

{\bfseries Critérios de Avaliação}

\headlinebox

\begin{itemize}
	\item A pontuação será baseada na rapidez com que o robô conseguir fazer o gol. O robô que marcar o gol mais rápido em uma de suas tentativas será o vencedor.
	\item Caso o robô não consiga realizar o gol, será considerada a distância mais próxima do gol que a bola alcançou.
	\item Se o robô não conseguir mover a bola, será considerado se ele tocou na bola como critério de avaliação.
\end{itemize}