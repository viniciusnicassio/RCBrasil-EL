\clearpage
\sffamily
{\bfseries\color[rgb]{0.4,0.4,0.4}Desafio de Controle - Corrida}
\phantomsection
\addcontentsline{toc}{subsection}{Corrida}

\bigskip

{\bfseries Objetivo}

\headlinebox

O objetivo deste desafio é medir a capacidade do robô de completar um percurso no menor tempo possível dentro de uma arena. Cada robô compete individualmente, e o tempo disponível para completar o percurso é limitado a 5 minutos.

\bigskip

{\bfseries Regras do Desafio}

\headlinebox

\begin{itemize}
	\item A corrida ocorre com um único robô na arena por vez.
	\item O robô deve iniciar a corrida posicionado inteiramente fora do campo de futebol, na \textbf{Área de Partida} (linha de gol).
	\item O tempo começa a ser contado assim que o robô toca as linhas laterais do campo. Se o robô já estiver posicionado na linha do gol, o tempo começa a contar no momento em que ele inicia o movimento.
	\item Se o robô não se mover após três (3) notificações do árbitro para largada, a corrida será encerrada sem movimento.
	\item Um robô completa a corrida quando qualquer parte dele, por mínima que seja, tocar a \textbf{Linha de Chegada} (meio-campo).
	\item O robô não é obrigado a terminar a corrida na posição vertical; é permitido que o robô pule. Se o robô cair, ele deve ser capaz de se levantar por conta própria.
	\item Se o robô não conseguir se levantar, ele poderá ser reposicionado no início do percurso, na \textbf{Área de Partida}. No entanto, qualquer toque de um membro da equipe no robô, seja ele estando de pé ou caído, exigirá que o robô seja reposicionado na \textbf{Área de Partida}, sem interrupção do cronômetro.
\end{itemize}

\bigskip

{\bfseries Critérios de Avaliação}

\headlinebox

\begin{itemize}
	\item O vencedor da corrida será o robô que completar o percurso no menor tempo.
	\item Se nenhum robô completar o percurso, o vencedor será determinado pelo robô que alcançar a posição mais distante da \textbf{Área de Partida}.
	\item Se a equipe optar por interromper a corrida antes do término, a posição em que o robô parou será considerada como sua posição final, sendo considerada a distância percorrida. O tempo decorrido será contabilizado como 3 minutos.
\end{itemize}

\bigskip

{\bfseries Tentativas Oficiais}

\headlinebox

\begin{itemize}
	\item Cada robô dispõe de 5 (cinco) tentativas na fase classificatória e 3 (três) tentativas nas finais para correr e registrar tempos oficiais.
	\item Deve haver um intervalo mínimo de 30 minutos entre as tentativas.
	\item O melhor resultado obtido pelo robô será considerado para fins de classificação e determinação dos campeões.
\end{itemize}