\clearpage
\sffamily
{\bfseries\color[rgb]{0.4,0.4,0.4}Desafios de Controle}
\phantomsection
\addcontentsline{toc}{subsection}{Desafios de Controle}

\bigskip

{\bfseries Introdução aos Desafios de Controle}

\headlinebox

Os desafios de controle têm como objetivo avaliar a habilidade dos robôs em realizar ações precisas e coordenadas em um ambiente de competição. Ao contrário dos desafios de visão, onde a detecção e análise visual são o foco principal, nos desafios de controle, a ênfase está na execução física e motora dos robôs. A capacidade de movimentação, precisão em chutes e a agilidade em completar tarefas dentro de limites de tempo são aspectos fundamentais que serão avaliados.

Nestes desafios, é imprescindível que as equipes utilizem um robô físico completo, equipado com os mecanismos necessários para realizar as tarefas propostas. A performance do robô será medida em termos de tempo, precisão e sucesso nas tarefas, sendo essenciais para a classificação e definição dos campeões.

\bigskip

{\bfseries Requisitos}

\headlinebox

\begin{itemize}
	\item \textbf{Presença de um Robô Físico Completo:} Diferentemente dos desafios de visão, nos desafios de controle é obrigatório o uso de um robô físico com todas as funcionalidades operacionais. O robô deve ser capaz de se movimentar, chutar e realizar as ações necessárias para completar os desafios.
	\item \textbf{Mecanismos de Movimentação e Execução:} O robô deve estar equipado com mecanismos adequados para a movimentação no campo, além de sistemas que permitam a execução precisa das tarefas, como chutar uma bola ou percorrer um percurso.
	\item \textbf{Autonomia Operacional:} O robô deve operar de forma autônoma durante os desafios, sem interferência direta da equipe, exceto nas situações previstas pelas regras, como reposicionamento inicial ou em caso de queda.
	\item \textbf{Tempo e Tentativas:} Cada robô terá um número limitado de tentativas para completar os desafios, e o desempenho será medido com base no tempo e na precisão das ações. As equipes devem estar preparadas para ajustar e otimizar o comportamento do robô entre as tentativas, respeitando os intervalos mínimos estabelecidos.
\end{itemize}