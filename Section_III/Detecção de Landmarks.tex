\clearpage
\sffamily
{\bfseries\color[rgb]{0.4,0.4,0.4}Desafio de Visão - Detecção de Landmarks}
\phantomsection
\addcontentsline{toc}{subsection}{Detecção de Landmarks}

\bigskip

{\bfseries Objetivo}

\headlinebox

Neste desafio, o robô será posicionado dentro do círculo central em qualquer posição escolhida pela equipe. O objetivo é que o robô detecte o máximo possível de landmarks no campo, como o círculo central, a marca do pênalti e as traves do gol.

\bigskip

{\bfseries Critérios de Avaliação}

\headlinebox

A pontuação será baseada na quantidade de landmarks distintos detectados pelo robô, com 10 pontos atribuídos por cada landmark válido. Caso um landmark distinto seja identificado, a equipe deve justificar a detecção ao juiz, que decidirá se a detecção é válida. Landmarks simétricos também serão contabilizados.

\bigskip

{\bfseries Regras Específicas}

\headlinebox

O robô poderá se movimentar pelo campo por um período de 2 minutos para tentar detectar mais landmarks e, assim, aumentar a pontuação. A posição inicial do robô dentro do círculo central pode ser escolhida livremente pela equipe, permitindo uma estratégia de detecção personalizada.