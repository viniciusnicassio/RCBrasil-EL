\clearpage
\sffamily
{\bfseries\color[rgb]{0.4,0.4,0.4}Desafio de Visão - Acompanhamento de Bola}
\phantomsection
\addcontentsline{toc}{subsection}{Acompanhamento de Bola}

\bigskip

{\bfseries Objetivo}

\headlinebox

Neste desafio, o robô será posicionado na linha do gol, voltado para o círculo central, devendo permanecer imóvel, com exceção da possibilidade de mover a cabeça para acompanhar uma bola em movimento. A câmera do robô deve manter a bola centralizada enquanto ela se move. A bola utilizada será a mesma da RoboCup Humanoid League e poderá ser consultada nas regras brasileiras da CBR.

\bigskip

{\bfseries Critérios de Avaliação}

\headlinebox

O objetivo é avaliar a distância máxima em que o robô consegue manter a bola centralizada em sua visão. Durante o desafio, o juiz poderá ocultar a bola para testar a capacidade de recuperação do robô. A distância máxima aceitável será aquela em que o robô conseguir recuperar a visão da bola após a desobstrução, dentro de um limite de 10 segundos. Movimentos paralelos à linha de campo poderão ser realizados para avaliação da visão, inclusive durante a oclusão da bola.

\bigskip

{\bfseries Regras Específicas}

\headlinebox

Durante o desafio, o robô não poderá se movimentar, sendo permitida apenas a movimentação da cabeça para acompanhar a bola em movimento.