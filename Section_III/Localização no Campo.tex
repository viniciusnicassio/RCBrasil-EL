\clearpage
\sffamily
{\bfseries\color[rgb]{0.4,0.4,0.4}Desafio de Visão - Localização no Campo}
\phantomsection
\addcontentsline{toc}{subsection}{Localização no Campo}

\bigskip

{\bfseries Objetivo}

\headlinebox

Neste desafio, o robô deve determinar sua posição no campo utilizando apenas movimentos da cabeça e giros no próprio eixo. O objetivo é avaliar a capacidade do robô de localizar sua posição com precisão dentro do campo.

\bigskip

{\bfseries Critérios de Avaliação}

\headlinebox

A precisão na determinação da posição do robô será avaliada com base na proximidade da posição identificada em relação a uma posição de referência conhecida. O sistema de pontuação será o seguinte:
\begin{itemize}
	\item \textbf{Dentro de 30 Centímetros da Posição de Referência:} 100 pontos
	\item \textbf{Entre 30 Centímetros e 1 Metro da Posição de Referência:} 50 pontos
	\item \textbf{Fora de 1 Metro da Posição de Referência:} 0 pontos
\end{itemize}

A ambiguidade na localização será considerada para desempate e para a determinação do ranking final das equipes.

\bigskip

{\bfseries Regras Específicas}

\headlinebox

O robô só poderá realizar movimentos da cabeça e giros no próprio eixo para determinar sua localização. Movimentos adicionais ou outros métodos de localização não serão permitidos.