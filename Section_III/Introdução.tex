\clearpage
\sffamily
{\bfseries\color[rgb]{0.4,0.4,0.4}Desafios de Visão}
\phantomsection
\addcontentsline{toc}{subsection}{Desafios de Visão}

\bigskip

{\bfseries Introdução aos Desafios de Visão}

\headlinebox

Os desafios de visão têm como objetivo avaliar a capacidade dos robôs de detectar objetos essenciais para o futebol de robôs. Para isso, as equipes podem utilizar uma ou duas câmeras (visão estéreo), sendo permitida a participação com robôs completos ou apenas com câmeras, desde que a altura especificada seja respeitada.

Durante os desafios, é permitido movimentar tanto a cabeça do robô quanto o próprio robô, conforme as necessidades do desafio. A câmera também pode ser movimentada, mas deve seguir as mesmas limitações de um operador humano.

Equipes que não possuírem um robô completo podem usar uma câmera em um tripé, desde que a altura mínima especificada nas dimensões do robô seja respeitada. Neste caso, a câmera deve permanecer fixa durante o desafio.

O processamento deve ser realizado no robô e o dispositivo não pode estar conectado à internet. As imagens capturadas pela câmera devem ser exibidas em uma tela para validação do desafio. Caso o processamento seja realizado no mesmo computador usado para visualização, haverá uma penalidade de 5\% na pontuação, exceto para equipes que utilizarem apenas a cabeça do robô e comprovarem que o processamento está sendo feito fora do computador de exibição.

\bigskip

{\bfseries Requisitos}

\headlinebox

Os requisitos básicos incluem um robô equipado com uma câmera e um monitor ou computador conectado para visualização. O computador deve ser usado exclusivamente para exibir as imagens, com todo o processamento realizado no robô. 

Para equipes que usarem uma câmera em um tripé, o processamento pode ser feito no mesmo computador de visualização, mas haverá uma penalidade de 5\% na pontuação. Equipes que utilizarem apenas a cabeça do robô e o módulo de processamento, sem a estrutura completa do corpo, não serão penalizadas, desde que o processamento seja comprovadamente feito fora do computador de exibição.