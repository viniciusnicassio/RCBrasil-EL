\clearpage
\sffamily
{\bfseries\color[rgb]{0.4,0.4,0.4}Desafio de Visão - Identificação de Robôs}
\phantomsection
\addcontentsline{toc}{subsection}{Identificação de Robôs}

\bigskip

{\bfseries Objetivo}

\headlinebox

Neste desafio, o robô da equipe será posicionado na linha do gol, voltado para o círculo central, devendo permanecer imóvel, exceto pela possibilidade de movimentar a cabeça. Serão posicionados seis robôs em posições aleatórias ao longo do campo, alternando entre posições antes e depois da linha central, começando pelo lado mais próximo do observador. Os robôs posicionados podem estar em pé ou sentados.

\bigskip

{\bfseries Critérios de Avaliação}

\headlinebox

O objetivo é avaliar quantos dos seis robôs diferentes o robô observador consegue identificar corretamente. O robô terá 10 segundos para recuperar a visão após qualquer obstrução. Se ocorrer uma falha, a detecção feita antes da obstrução não será contabilizada. Serão concedidos 10 pontos por cada detecção correta. Cada equipe participante deve disponibilizar um robô para este desafio; esse robô não precisa estar ligado ou operacional, servindo apenas como modelo para detecção. A pontuação máxima será determinada no dia da competição, de acordo com a quantidade de robôs distintos disponíveis.

\bigskip

{\bfseries Regras Específicas}

\headlinebox

Durante o desafio, o robô observador não poderá se movimentar, sendo permitida apenas a movimentação da cabeça.